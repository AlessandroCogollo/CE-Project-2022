\documentclass[titlepage,a4paper]{article}

% Escritura de acentos:
%--------------------------------------------------------------------------

\usepackage[utf8]{inputenc}
\usepackage[T1]{fontenc}

% Selección de idioma (español)
%--------------------------------------------------------------------------


% Título, autor, materia y fecha (rellenar)
%--------------------------------------------------------------------------

\newcommand{\titulo}{Informe} 

\newcommand{\facultad}{Ciencias} % ej.: Ciencias, Biología, Química...

\newcommand{\autor}{Alessandro Cogollo}
%\newcommand{\autor2}{Autor2}
% etc

\newcommand{\materia}{Materia}

\newcommand{\ciudad}{Oviedo}

\newcommand{\fecha}{\today} % Muestra por defecto la fecha actual

% Paquete para generar texto de relleno. Se puede eliminar
%--------------------------------------------------------------------------

\usepackage{lipsum}

% Dimensiones de página
%--------------------------------------------------------------------------

\usepackage[a4paper,includeheadfoot,margin=2.54cm]{geometry}

% Paquetes de la AMS
%--------------------------------------------------------------------------

\usepackage{amsmath}
\usepackage{amsfonts}
\usepackage{amssymb}
\usepackage{amsthm}


% Paquetes de símbolos usuales y de formato de ecuaciones
%--------------------------------------------------------------------------

\usepackage{textcomp}
\usepackage{mathtools}
\usepackage{commath}

% Teoremas
%--------------------------------------------------------------------------

\newtheorem{thm}{Teorema}
\newtheorem*{thm*}{Teorema}
\newtheorem{cor}[thm]{Corolario}
\newtheorem{lem}[thm]{Lema}
\newtheorem{prop}[thm]{Proposición}
\theoremstyle{definition}
\newtheorem{defn}[thm]{Definición}
\newtheorem*{defn*}{Definición}
\theoremstyle{remark}
\newtheorem{rem}[thm]{Observación}



% Atajos.
% Se pueden definir comandos nuevos para acortar cosas que se usan
% frecuentemente. Como ejemplo, aquí se definen las letras dobles usadas para
% los conjuntos de los números naturales, enteros, racionales, reales
% y complejos.
%--------------------------------------------------------------------------

\def\NN{\mathbb{N}}
\def\ZZ{\mathbb{Z}}
\def\QQ{\mathbb{Q}}
\def\RR{\mathbb{R}}
\def\CC{\mathbb{C}}

% Paquetes para la inserción de imágenes y figuras
%--------------------------------------------------------------------------

\usepackage{float}
\usepackage{graphicx}
\usepackage{epstopdf}
\usepackage{caption}

% Formato de los índices
%--------------------------------------------------------------------------
\usepackage{hyperref} % referencias interactivas
\usepackage{tocloft} % paquete de diseño tipográfico

\renewcommand{\cftsecfont}{\normalfont} % evita que los títulos del índice aparezcan en negrita
\renewcommand{\cftsecleader}{\cftdotfill{\cftdotsep}} % crea una línea de puntos entre el índice y el número

% Paquetes y configuración para la inserción de código
% Por defecto los colores emulan los del editor de MATLAB
%--------------------------------------------------------------------------

\usepackage{listings} % paquete para introducir los "scripts"

\lstset{language=Octave} % lenguaje a utilizar (ej.: Octave, R, Python...)

\usepackage{color} % paquete de color y definición de los colores a usar

\definecolor{mygreen}{rgb}{0,0.6,0}
\definecolor{mygray}{rgb}{0.5,0.5,0.5}
\definecolor{mymauve}{rgb}{0.58,0,0.82}

\lstset{ % 
	backgroundcolor=\color{white},   % color de fondo
	basicstyle=\footnotesize,        % tamaño de fuente del código
	breaklines=true,                 % salto de línea automático
	captionpos=t,                    % coloca los títulos o comentarios sobre el código
	commentstyle=\color{mygreen},    % formato de los comentarios
	escapeinside={\%*}{*)},          % permite añadir LaTeX dentro del código
	keywordstyle=\color{blue},       % formato de destacado de palabras clave
	%otherkeywords={}				 % permite definir nuevas palabras clave
	%deletekeywords={}				 % permite evitar que algunas palabras se resalten
	stringstyle=\color{mymauve},     % formato de los "strings"
	frame=tb,						 % coloca una línea sobre el código y otra por debajo
	tabsize=3,						 % tamaño de las tabulaciones
    showstringspaces=false,			 % evita que se destaquen los espacios en los "strings"
	upquote=true					 % comillas rectas
}


\lstset{literate= % permite utilizar tildes y otros símbolos dentro del código
	{á}{{\'a}}1 {é}{{\'e}}1 {í}{{\'i}}1 {ó}{{\'o}}1 {ú}{{\'u}}1
	{Á}{{\'A}}1 {É}{{\'E}}1 {Í}{{\'I}}1 {Ó}{{\'O}}1 {Ú}{{\'U}}1
	{à}{{\`a}}1 {è}{{\`e}}1 {ì}{{\`i}}1 {ò}{{\`o}}1 {ù}{{\`u}}1
	{À}{{\`A}}1 {È}{{\'E}}1 {Ì}{{\`I}}1 {Ò}{{\`O}}1 {Ù}{{\`U}}1
	{ä}{{\"a}}1 {ë}{{\"e}}1 {ï}{{\"i}}1 {ö}{{\"o}}1 {ü}{{\"u}}1
	{Ä}{{\"A}}1 {Ë}{{\"E}}1 {Ï}{{\"I}}1 {Ö}{{\"O}}1 {Ü}{{\"U}}1
	{â}{{\^a}}1 {ê}{{\^e}}1 {î}{{\^i}}1 {ô}{{\^o}}1 {û}{{\^u}}1
	{Â}{{\^A}}1 {Ê}{{\^E}}1 {Î}{{\^I}}1 {Ô}{{\^O}}1 {Û}{{\^U}}1
	{œ}{{\oe}}1 {Œ}{{\OE}}1 {æ}{{\ae}}1 {Æ}{{\AE}}1 {ß}{{\ss}}1
	{ű}{{\H{u}}}1 {Ű}{{\H{U}}}1 {ő}{{\H{o}}}1 {Ő}{{\H{O}}}1
	{ç}{{\c c}}1 {Ç}{{\c C}}1 {ø}{{\o}}1 {å}{{\r a}}1 {Å}{{\r A}}1
	{€}{{\EUR}}1 {£}{{\pounds}}1 {~}{{$\sim$}}1
}


% Encabezado y pie de página (no es necesaria modificación)
%--------------------------------------------------------------------------

\usepackage{fancyhdr} % Paquete de estilo 

\pagestyle{fancy} % selección del estilo
\fancyhf{}

\renewcommand{\headrulewidth}{0.5pt} % Grosor de las líneas
\renewcommand{\footrulewidth}{0.5pt}

\lhead{\scshape \materia} 
\rhead{\titulo}
\lfoot{\autor}
\rfoot{\thepage}

% Inicio del documento
%--------------------------------------------------------------------------

\begin{document}
% Si no se desea portada, sustitúyanse las siguientes líneas por:
%\title{\titulo}
%\author{\autor}
%\date{\fecha}
	
	% Portada
	%----------------------------------------------------------------------
	
	\begin{titlepage}
		
		\centering	
		
		\includegraphics[width=0.6\textwidth]{escudo.jpg}\par\vspace{0.5cm}
		
		{\scshape\LARGE Facultad de \facultad\par}
		
		\vspace{1cm}
		
		{\scshape\Large \materia\par}
		
		\vspace{1.5cm}
		
		{\huge\bfseries \titulo\par}	% puede ser necesario modificar si se desea más de una línea
		
		\vspace{3cm}
		
		{\Large\itshape \autor} %\par\vspace{0.1cm} \autor2\par\vspace{0.1cm} \autor3}
		
		\vfill
		
		\ciudad\par
		
		\fecha
		
	\end{titlepage}
	
	% Dejamos la siguiente página en blanco y sin numerar:
	
	\clearpage\thispagestyle{empty}\mbox{}\setcounter{page}{0}\clearpage
    
  % Abstract (borrar el "%" si se requiere)
  %----------------------------------------------------------------------
  %\abstract{}
   
  % Índice
  %----------------------------------------------------------------------
  \tableofcontents{}
  \newpage{} % elimínese si no se desea un salto de página tras el índice
	
\section{Introducción}

\lipsum[1-1] % texto de relleno

\begin{equation}
E=m \cdot c^2
\end{equation}

\section{Fundamento}

\lipsum[2-2]

\begin{thm}
Sean $\tilde{K}:K$ una extensión finita normal de cuerpos y sea $L$ un cuerpo intermedio ($K\subseteq L\subseteq \tilde{K}$). Todo $K$-monomorfismo $g:L\longrightarrow \tilde{K}$ se extiende a un $K$-automorfismo $h:\tilde{K}\longrightarrow\tilde{K}$ ($h_{\mid L}=g$).
\end{thm}

\lipsum[3-3]

\begin{equation}
\begin{split}
y_{1}=&y_{0}+h\cdot f(t_{0},y_{0}) \\
y_{2}=&y_{1}+h\cdot f(t_{1},y_{1}) \\
	  &\vdots \\
y_{n}=&y_{n-1}+h\cdot f(t_{n-1},y_{n-1})
\end{split}
\end{equation}

\section{Procedimiento}

\lipsum[4-4]

\begin{itemize}
\item Esto.
\item Lo otro.
\item Lo de más allá.
\end{itemize}

\begin{lstlisting}[title=hello\_world.m]
n=10;
for i=1:n
	printf("Hello world!\n")
end
\end{lstlisting}

\section{Conclusiones}

\lipsum[5-5]

\begin{equation}
\left[-\frac{\hbar}{2m}\nabla^2 + U(\vec{r}) \right]\Psi(\vec{r}) = E\Psi(\vec{r})
\end{equation}

\section{Bibliografía}

\indent

\leftskip 0.3in
\parindent -0.3in

\textsc{Cicerón}, Marcus Tullius, \textit{De finibus bonorum et malorum}, sección 1.10.33, Roma, 45 a.C. [en línea. Fecha de consulta: 10 de marzo de 2017] Disponible en: <http://www.thelatinlibrary.com/cicero/fin.shtml> 

\end{document}


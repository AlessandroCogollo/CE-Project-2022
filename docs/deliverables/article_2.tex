%%%%%%%%%%%%%%%%%%%%%%%%%%%%%%%%%%%%%%%%%
% Journal Article
% LaTeX Template
% Version 1.4 (15/5/16)
%
% This template has been downloaded from:
% http://www.LaTeXTemplates.com
%
% Original author:
% Frits Wenneker (http://www.howtotex.com) with extensive modifications by
% Vel (vel@LaTeXTemplates.com)
%
% License:
% CC BY-NC-SA 3.0 (http://creativecommons.org/licenses/by-nc-sa/3.0/)
%
%%%%%%%%%%%%%%%%%%%%%%%%%%%%%%%%%%%%%%%%%

%----------------------------------------------------------------------------------------
%	PACKAGES AND OTHER DOCUMENT CONFIGURATIONS
%----------------------------------------------------------------------------------------

\documentclass[twoside,twocolumn]{article}

\usepackage{blindtext} % Package to generate dummy text throughout this template 

\usepackage[sc]{mathpazo} % Use the Palatino font
\usepackage[T1]{fontenc} % Use 8-bit encoding that has 256 glyphs
\linespread{1.05} % Line spacing - Palatino needs more space between lines
\usepackage{microtype} % Slightly tweak font spacing for aesthetics

\usepackage[english]{babel} % Language hyphenation and typographical rules

\usepackage[hmarginratio=1:1,top=32mm,columnsep=20pt]{geometry} % Document margins
\usepackage[hang, small,labelfont=bf,up,textfont=it,up]{caption} % Custom captions under/above floats in tables or figures
\usepackage{booktabs} % Horizontal rules in tables

\usepackage{lettrine} % The lettrine is the first enlarged letter at the beginning of the text

\usepackage{enumitem} % Customized lists
\setlist[itemize]{noitemsep} % Make itemize lists more compact

\usepackage{abstract} % Allows abstract customization
\renewcommand{\abstractnamefont}{\normalfont\bfseries} % Set the "Abstract" text to bold
\renewcommand{\abstracttextfont}{\normalfont\small\itshape} % Set the abstract itself to small italic text

\usepackage{titlesec} % Allows customization of titles
\renewcommand\thesection{\Roman{section}} % Roman numerals for the sections
\renewcommand\thesubsection{\roman{subsection}} % roman numerals for subsections
\titleformat{\section}[block]{\large\scshape\centering}{\thesection.}{1em}{} % Change the look of the section titles
\titleformat{\subsection}[block]{\large}{\thesubsection.}{1em}{} % Change the look of the section titles

\usepackage{fancyhdr} % Headers and footers
\pagestyle{fancy} % All pages have headers and footers
\fancyhead{} % Blank out the default header
\fancyfoot{} % Blank out the default footer
\fancyhead[C]{Running title $\bullet$ May 2016 $\bullet$ Vol. XXI, No. 1} % Custom header text
\fancyfoot[RO,LE]{\thepage} % Custom footer text

\usepackage{titling} % Customizing the title section

\usepackage{hyperref} % For hyperlinks in the PDF

%----------------------------------------------------------------------------------------
%	TITLE SECTION
%----------------------------------------------------------------------------------------

\setlength{\droptitle}{-4\baselineskip} % Move the title up

\pretitle{\begin{center}\Huge\bfseries} % Article title formatting
\posttitle{\end{center}} % Article title closing formatting
\title{Renewable Energy\\Development Scenarios} % Article title
\author{%
\textsc{Alessandro Cogollo}\thanks{A thank you or further information} \\[1ex] % Your name
\normalsize Politecnico di Milano \\ % Your institution
\normalsize \href{mailto:cogolloalessandro@gmail.com}{cogolloalessandro@gmail.com} % Your email address
%\and % Uncomment if 2 authors are required, duplicate these 4 lines if more
%\textsc{Jane Smith}\thanks{Corresponding author} \\[1ex] % Second author's name
%\normalsize University of Utah \\ % Second author's institution
%\normalsize \href{mailto:jane@smith.com}{jane@smith.com} % Second author's email address
}
\date{\today} % Leave empty to omit a date
\renewcommand{\maketitlehookd}{%
\begin{abstract}
\noindent This manual aims to describe the software developed as a Computer Engineering project, presenting technologies, tools and architectural choices based on the scenario in which it's located. The developed software comes in hand to support the renewable enery development, providing futures scenarios used to lead changes. Furthermore, the manual represents a brief guide with an explanation on how to install the software.
\end{abstract}
}

%----------------------------------------------------------------------------------------

\begin{document}

% Print the title
\maketitle

%----------------------------------------------------------------------------------------
%	ARTICLE CONTENTS
%----------------------------------------------------------------------------------------

\section{Introduction}

\lettrine[nindent=0em,lines=3]{T} he proposed project aims to create an interactive web interface which will be integrated in the RSE S.p.A. dissemination platforms, for supporting future renewable energy development scenarios, which constitutes one of the core activities in RSE.
The website will support the computation of the photovoltaic capacity distribution and the expected production in Italy at a province scale for the achievement of the European Green Deal goals. Starting from variable input parameters and spatialized indicators, the interface should allow the spatial and graphical representation of the intermediate and final outputs. 

%------------------------------------------------

\section{Architecture}

The software is meant for web, and is composed by three main modules, allowing any distribution on multiple servers. First module is the web server, which receives requests from a client and dispatch them to the computation module; it's deputed to retrieving parameters inserted by the user, and to display the plotted graphs. Computational module is responsible for querying the database (PostgreSQL, explained later) and obtaining input values to calculate arrays, required to plot graphs and maps. Finally, the computational module passes values to be plotted to the graph module, which displays them, and return to the web server the map as an html file. 
Pictures explains

%-----------------------------------------------

\section{Tools}

The software is composed by three main modules, to allow any distribution on multiple servers. First module is the web server, which receives requests from a client used to calculate the parameters

Text requiring further explanation\footnote{Example footnote}.

%-----------------------------------------------

\section{Datas}
Regarding the database, a PostgreSQL database was choosen, since it represents a powerful, open source object-relational database system, and has powerful add-ons such as the popular PostGIS geospatial database extender. In fact, PostgreSQL offers in addition to the usual MySQL datatypes, the possibility to store geometry types: Point, Line, Circle, Polygon, which has been used to store datas about provinces' borders.

Accenni a dati ISTAT
Accenni a setup del db

\begin{table}
	\caption{Input}
	\centering
	\begin{tabular}{llr}
		\toprule
		\multicolumn{2}{c}{Name} \\
		\cmidrule(r){1-2}
		Table & Input\\
		\midrule
		Photovoltaic Installed & Installed Power LAND\\
		Photovoltaic Installed & Installed Power ROOF\\
		\bottomrule
	\end{tabular}
\end{table}
		
%------------------------------------------------

\section{Parameters}


%------------------------------------------------

\section{Formulas}

\begin{equation}
\label{eq:emc}
sum built surface = sum values of \textbf{built surface}
\end{equation}

\blindtext % Dummy text

%------------------------------------------------

\section{Flow}

\subsection{Subsection One}

A statement requiring citation \cite{Figueredo:2009dg}.
\blindtext % Dummy text

\subsection{Subsection Two}

\blindtext % Dummy text

\section{Visualization}

\subsection{Graphs}

A statement requiring citation \cite{Figueredo:2009dg}.
\blindtext % Dummy text

\subsection{Plots}

\blindtext % Dummy text

%--------------------------------------------------------------

\section{Installation}
Provide a db dump

%--------------------------------------------------------------

\section{Conclusion}

%----------------------------------------------------------------------------------------
%	REFERENCE LIST
%----------------------------------------------------------------------------------------

\begin{thebibliography}{99} % Bibliography - this is intentionally simple in this template

\bibitem[Figueredo and Wolf, 2009]{Figueredo:2009dg}
Figueredo, A.~J. and Wolf, P. S.~A. (2009).
\newblock Assortative pairing and life history strategy - a cross-cultural
  study.
\newblock {\em Human Nature}, 20:317--330.
 
\end{thebibliography}

%----------------------------------------------------------------------------------------

\end{document}
